%%%%%%%%%%%%%%%%%%%%%%%%%%%%%%%%%%%%%%%%%%%%%%%%%%%%%%%%%%%%%%%%%%%%%%%%%%%%%%%%
%2345678901234567890123456789012345678901234567890123456789012345678901234567890
%        1         2         3         4         5         6         7         8

\documentclass[letterpaper, 10 pt, conference]{ieeeconf}  % Comment this line out
                                                          % if you need a4paper
%\documentclass[a4paper, 10pt, conference]{ieeeconf}      % Use this line for a4
                                                          % paper

\IEEEoverridecommandlockouts                              % This command is only
                                                          % needed if you want to
                                                          % use the \thanks command
\overrideIEEEmargins
% See the \addtolength command later in the file to balance the column lengths
% on the last page of the document



% The following packages can be found on http:\\www.ctan.org
%\usepackage{graphics} % for pdf, bitmapped graphics files
%\usepackage{epsfig} % for postscript graphics files
%\usepackage{mathptmx} % assumes new font selection scheme installed
%\usepackage{times} % assumes new font selection scheme installed
%\usepackage{amsmath} % assumes amsmath package installed
%\usepackage{amssymb}  % assumes amsmath package installed

\title{\LARGE \bf
Towards 3D Object Recognition via Classification of Arbitrary Object Tracks
}

%\author{ \parbox{3 in}{\centering Huibert Kwakernaak*
%         \thanks{*Use the $\backslash$thanks command to put information here}\\
%         Faculty of Electrical Engineering, Mathematics and Computer Science\\
%         University of Twente\\
%         7500 AE Enschede, The Netherlands\\
%         {\tt\small h.kwakernaak@autsubmit.com}}
%         \hspace*{ 0.5 in}
%         \parbox{3 in}{ \centering Pradeep Misra**
%         \thanks{**The footnote marks may be inserted manually}\\
%        Department of Electrical Engineering \\
%         Wright State University\\
%         Dayton, OH 45435, USA\\
%         {\tt\small pmisra@cs.wright.edu}}
%}

\author{Huibert Kwakernaak and Pradeep Misra% <-this % stops a space
\thanks{This work was not supported by any organization}% <-this % stops a space
\thanks{H. Kwakernaak is with Faculty of Electrical Engineering, Mathematics and Computer Science,
        University of Twente, 7500 AE Enschede, The Netherlands
        {\tt\small h.kwakernaak@autsubmit.com}}%
\thanks{P. Misra is with the Department of Electrical Engineering, Wright State University,
        Dayton, OH 45435, USA
        {\tt\small pmisra@cs.wright.edu}}%
}


\begin{document}



\maketitle
\thispagestyle{empty}
\pagestyle{empty}


%%%%%%%%%%%%%%%%%%%%%%%%%%%%%%%%%%%%%%%%%%%%%%%%%%%%%%%%%%%%%%%%%%%%%%%%%%%%%%%%
\begin{abstract}

These instructions provide basic guidelines for preparing camera-ready (CR)
Proceedings-style papers. This document is itself an example of the
desired layout for CR papers (inclusive of this abstract). The document
contains information regarding desktop publishing format, type sizes, and
type faces. Style rules are provided that explain how to handle equations,
units, figures, tables, references, abbreviations, and acronyms. Sections
are also devoted to the preparation of the references and acknowledgments.

\end{abstract}


%%%%%%%%%%%%%%%%%%%%%%%%%%%%%%%%%%%%%%%%%%%%%%%%%%%%%%%%%%%%%%%%%%%%%%%%%%%%%%%%
\section{INTRODUCTION}

Your goal is to simulate, as closely as possible, the usual appearance of typeset
 papers. This document provides an example of the desired layout and contains
 information regarding desktop publishing format, type sizes, and type faces.

\subsection{Full-Size Camera-Ready (CR) Copy}

If you have desktop publishing facilities, (the use of a computer to aid
 in the assembly of words and illustrations on pages) prepare your CR paper
  in full-size format, on paper 21.6 x 27.9 cm (8.5 x 11 in or 51 x 66 picas).
  It must be output on a printer (e.g., laser printer) having 300 dots/in, or
  better, resolution. Lesser quality printers, such as dot matrix printers,
   are not acceptable, as the manuscript will not reproduce the desired quality.

\subsubsection{Typefaces and Sizes:} There are many different typefaces and a large
variety of fonts (a complete set of characters in the same typeface, style,
 and size). Please use a proportional serif typeface such as Times Roman,
 or Dutch. If these are not available to you, use the closest typeface you
  can. The minimum typesize for the body of the text is 10 point. The minimum
  size for applications like table captions, footnotes, and text subscripts
  is 8 point. As an aid in gauging type size, 1 point is about 0.35 mm (1/72in).
   Examples are as follows:

\subsubsection{Format:} In formatting your original 8.5" x 11" page, set top and
bottom margins to 25 mm (1 in or 6 picas), and left and right margins
to about 18 mm (0.7 in or 4 picas). The column width is 88 mm (3.5 in or 21 picas).
 The space between the two columns is 5 mm(0.2 in or 1 pica). Paragraph
 indentation is about 3.5 mm (0.14 in or 1 pica). Left- and right-justify your
 columns. Cut A4 papers to 28 cm. Use either one or two spaces between sections,
 and between text and tables or figures, to adjust the column length.
  On the last page of your paper, try to adjust the lengths of the
  two-columns so that they are the same. Use automatic hyphenation and
   check spelling. Either digitize or paste your figures.

\begin{table}
\caption{An Example of a Table}
\label{table_example}
\begin{center}
\begin{tabular}{|c||c|}
\hline
One & Two\\
\hline
Three & Four\\
\hline
\end{tabular}
\end{center}
\end{table}


%%%%%%%%%%%%%%%%%%%%%%%%%%%%%%%%%%%%%%%%%%%%%%%%%%%%%%%%%%%%%%%%%%%%%%%%%%%%%%%%
\section{UNITS}

Metric units are preferred for use in IEEE publications in light of their
international readership and the inherent convenience of these units in many fields.
In particular, the use of the International System of Units (SI Units) is advocated.
 This system includes a subsystem the MKSA units, which are based on the
 meter, kilogram, second, and ampere. British units may be used as secondary units
 (in parenthesis). An exception is when British units are used as identifiers in trade,
 such as, 3.5 inch disk drive.


\addtolength{\textheight}{-3cm}   % This command serves to balance the column lengths
                                  % on the last page of the document manually. It shortens
                                  % the textheight of the last page by a suitable amount.
                                  % This command does not take effect until the next page
                                  % so it should come on the page before the last. Make
                                  % sure that you do not shorten the textheight too much.

%%%%%%%%%%%%%%%%%%%%%%%%%%%%%%%%%%%%%%%%%%%%%%%%%%%%%%%%%%%%%%%%%%%%%%%%%%%%%%%%
\section{ADDITIONAL REQUIREMENTS}

\subsection{Figures and Tables}

Position figures and tables at the tops and bottoms of columns.
Avoid placing them in the middle of columns. Large figures and tables
may span across both columns. Figure captions should be below the figures;
 table captions should be above the tables. Avoid placing figures and tables
  before their first mention in the text. Use the abbreviation ``Fig. 1'',
  even at the beginning of a sentence.
Figure axis labels are often a source of confusion.
Try to use words rather then symbols. As an example write the quantity ``Inductance",
 or ``Inductance L'', not just.
 Put units in parentheses. Do not label axes only with units.
 In the example, write ``Inductance (mH)'', or ``Inductance L (mH)'', not just ``mH''.
 Do not label axes with the ratio of quantities and units.
 For example, write ``Temperature (K)'', not ``Temperature/K''.

\subsection{Numbering}

Number reference citations consecutively in square brackets \cite{c1}.
 The sentence punctuation follows the brackets \cite{c2}.
 Refer simply to the reference number, as in \cite{c3}.
 Do not use ``ref. \cite{c3}'' or ``reference \cite{c3}''.
Number footnotes separately in superscripts\footnote{This is a footnote}
Place the actual footnote at the bottom of the column in which it is cited.
Do not put footnotes in the reference list.
Use letters for table footnotes (see Table I).

\subsection{Abbreviations and Acronyms}

Define abbreviations and acronyms the first time they are used in the text,
even after they have been defined in the abstract. Abbreviations such as
IEEE, SI, CGS, ac, dc, and rms do not have to be defined. Do not use
abbreviations in the title unless they are unavoidable.

\subsection{Equations}

Number equations consecutively with equation numbers in parentheses flush
 with the right margin, as in (1). To make your equations more compact
 you may use the solidus (/), the exp. function, or appropriate exponents.
  Italicize Roman symbols for quantities and variables, but not Greek symbols.
   Use a long dash rather then hyphen for a minus sign. Use parentheses to avoid
    ambiguities in the denominator.
Punctuate equations with commas or periods when they are part of a sentence:
$$\Gamma_2 a^2 + \Gamma_3 a^3 + \Gamma_4 a^4 + ... = \lambda \Lambda(x),$$
where $\lambda$ is an auxiliary parameter.

Be sure that the symbols in your equation have been defined before the
equation appears or immediately following.
Use ``(1),'' not ``Eq. (1)'' or ``Equation (1),''
except at the beginning of a sentence: ``Equation (1) is ...''.

   \begin{figure}[thpb]
      \centering
      %\includegraphics[scale=1.0]{figurefile}
      \caption{Inductance of oscillation winding on amorphous
       magnetic core versus DC bias magnetic field}
      \label{figurelabel}
   \end{figure}

%%%%%%%%%%%%%%%%%%%%%%%%%%%%%%%%%%%%%%%%%%%%%%%%%%%%%%%%%%%%%%%%%%%%%%%%%%%%%%%%
\section{CONCLUSIONS AND FUTURE WORKS}

\subsection{Conclusions}

This is a repeat.
Position figures and tables at the tops and bottoms of columns.
Avoid placing them in the middle of columns. Large figures and tables
may span across both columns. Figure captions should be below the figures;
 table captions should be above the tables. Avoid placing figures and tables
  before their first mention in the text. Use the abbreviation ``Fig. 1'',
  even at the beginning of a sentence.
Figure axis labels are often a source of confusion.
Try to use words rather then symbols. As an example write the quantity ``Inductance",
 or ``Inductance L'', not just.
 Put units in parentheses. Do not label axes only with units.
 In the example, write ``Inductance (mH)'', or ``Inductance L (mH)'', not just ``mH''.
 Do not label axes with the ratio of quantities and units.
 For example, write ``Temperature (K)'', not ``Temperature/K''.


\subsection{Future Works}

This is a repeat.
Position figures and tables at the tops and bottoms of columns.
Avoid placing them in the middle of columns. Large figures and tables
may span across both columns. Figure captions should be below the figures;
 table captions should be above the tables. Avoid placing figures and tables
  before their first mention in the text. Use the abbreviation ``Fig. 1'',
  even at the beginning of a sentence.
Figure axis labels are often a source of confusion.
Try to use words rather then symbols. As an example write the quantity ``Inductance",
 or ``Inductance L'', not just.
 Put units in parentheses. Do not label axes only with units.
 In the example, write ``Inductance (mH)'', or ``Inductance L (mH)'', not just ``mH''.
 Do not label axes with the ratio of quantities and units.
 For example, write ``Temperature (K)'', not ``Temperature/K''.

%%%%%%%%%%%%%%%%%%%%%%%%%%%%%%%%%%%%%%%%%%%%%%%%%%%%%%%%%%%%%%%%%%%%%%%%%%%%%%%%
\section{ACKNOWLEDGMENTS}

The authors gratefully acknowledge the contribution of National Research Organization and reviewers' comments.


%%%%%%%%%%%%%%%%%%%%%%%%%%%%%%%%%%%%%%%%%%%%%%%%%%%%%%%%%%%%%%%%%%%%%%%%%%%%%%%%

References are important to the reader; therefore, each citation must be complete and correct. If at all possible, references should be commonly available publications.

\begin{thebibliography}{99}

\bibitem{c1}
J.G.F. Francis, The QR Transformation I, {\it Comput. J.}, vol. 4, 1961, pp 265-271.

\bibitem{c2}
H. Kwakernaak and R. Sivan, {\it Modern Signals and Systems}, Prentice Hall, Englewood Cliffs, NJ; 1991.

\bibitem{c3}
D. Boley and R. Maier, "A Parallel QR Algorithm for the Non-Symmetric Eigenvalue Algorithm", {\it in Third SIAM Conference on Applied Linear Algebra}, Madison, WI, 1988, pp. A20.

\end{thebibliography}

\end{document}
